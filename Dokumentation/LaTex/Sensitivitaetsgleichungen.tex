\documentclass[12pt, a4paper]{scrartcl}
\usepackage{fullpage}
\usepackage[utf8]{inputenc} %Zeichensatzkodierung (auch Sonderzeichen -> ä,ö,ü)
\usepackage[english, ngerman]{babel} %Sprache -> Silbentrennung, automatisch generierte Texte auf deutsch; ngerman -> NEUE deutsche Rechtschreibung
\usepackage[T1]{fontenc} %Sonderzeichen allg
\usepackage{lmodern} %PDF-optimierte Schrift

\usepackage{amsmath, amssymb, amsbsy, amsthm} %Mathepakete ; amsthm: Theorem-Umgebung; muss nach amsmath eingebunden werden

\usepackage{xcolor} %bunte Schrift 

\setlength{\parindent}{0pt}

\newcommand{\R}{\mathbb{R}}


\begin{document}
Sei $x \in \mathbb{R}^{k \cdot 13}$, $u \in \mathbb{R}^{k \cdot 4}$, $f(x, u) \in \mathbb{R}^{k \cdot 13}$ und sei zu dem $f \in C^{\infty}$ Dann gibt es für die ODE $\dot x = f(x, u)$, weg der Lipschitzstetigkeit, nach Poincare eine eindeutige Lösung $\phi(x, u) \in \mathbb{R}^{k \cdot 13}$ mit 
\begin{align*}
 	\dot \phi(x, u) = f(\phi(x, u), u)
\end{align*}
Man betrachte nun die Differenzialgleichung zum Zeitpunkt $k$, dann folgt für die Differenzierung nach der $x_i$ Komponente
\begin{align*}
	\frac{d}{dx^k_i} \frac{d}{dt^k} \phi(x^k, u^k) = \frac{d}{dx^k_i} f(\phi(x^k, u^k), u^k) \\
\end{align*}
Da $f \in C^{\infty}$ folgt das die Gleichung Sensitivität ist und die Ableitungen vertauscht werden können.
\begin{align*}
\frac{d}{dt^k} \frac{d}{dx^k_i} \phi(x^k, u^k) = \frac{d}{dx^k_i} f(\phi(x^k, u^k), u^k)  &= \frac{d}{dx^k} \left[f(\phi(x^k, u^k), u^k) \right] \cdot \frac{d}{dx^k_i}\phi(x^k, u^k) \\
&+ \frac{d}{du^k}\left[ f(\phi(x^k, u^k), u^k)  \right] \cdot \underbrace{\frac{du^k}{dx^k_i}}_{=0}
\end{align*}
mit $M^k = \frac{d}{dx^k_i} \phi(x^k, u^k) \in \mathbb{R}^{13 \times 13}, i = 1..13$ folgt die ODE  
\begin{align*}
\frac{d}{dt^k} M^k &= \underbrace{\frac{d}{dx^k} \left[f(\phi(x^k, u^k), u^k) \right]}_{\in \mathbb{R}^{13 \times 13}} \cdot M^k
\end{align*}
mit der Anfangsbedingung $M_0^k = I \in \mathbb{R}^{13 \times 13} $ \\\\
Analog folgt für die Differenzierung nach der $u_i$ Komponente
\begin{align*}
\frac{d}{dt^k} \frac{d}{du^k_i} \phi(x^k, u^k) &= \frac{d}{dx^k} \left[f(\phi(x^k, u^k), u^k) \right] \cdot \frac{d}{du^k_i}\phi(x^k, u^k) \\
&+ \frac{d}{du^k}\left[ f(\phi(x^k, u^k), u^k)  \right] \cdot \frac{du^k}{du^k_i}
\end{align*}
mit $N^k = \frac{d}{du^k_i} \phi(x^k, u^k) \in \mathbb{R}^{13 \times 4}, i = 1..4$ folgt die ODE 
\begin{align*}
\frac{d}{dt^k} N^k &= \underbrace{\frac{d}{dx^k} \left[f(\phi(x^k, u^k), u^k) \right]}_{\in \mathbb{R}^{13 \times 13}} \cdot N^k \\
&+ \underbrace{
	\frac{d}{du^k}
		\left[ f(\phi(x^k, u^k), u^k)  \right]
		   }_
{\in \mathbb{R}^{13 \times 4}}
		   \cdot \underbrace{I^k}_{\in \mathbb{R}^{4 \times 4}}
\end{align*}
mit folgender Anfangsbedingung $N_0^k = 0 \in \mathbb{R}^{13 \times 4}$
\end{document}