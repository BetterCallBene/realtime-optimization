\documentclass[12pt, a4paper]{scrartcl}
\usepackage{fullpage}
\usepackage[utf8]{inputenc} %Zeichensatzkodierung (auch Sonderzeichen -> ä,ö,ü)
\usepackage[english, ngerman]{babel} %Sprache -> Silbentrennung, automatisch generierte Texte auf deutsch; ngerman -> NEUE deutsche Rechtschreibung
\usepackage[T1]{fontenc} %Sonderzeichen allg
\usepackage{lmodern} %PDF-optimierte Schrift

\usepackage{amsmath, amssymb, amsbsy, amsthm} %Mathepakete ; amsthm: Theorem-Umgebung; muss nach amsmath eingebunden werden

\usepackage{xcolor} %bunte Schrift 

\setlength{\parindent}{0pt}
%\allowdisplaybreaks[1] % erlaubt Zeilenumbrüche in zB. align-Umgebung 

\newcommand{\R}{\mathbb{R}}


\begin{document}
\subsection*{Jacobimatrix}
Sei $x \in \mathbb{R}^{k \cdot 13}$, $u \in \mathbb{R}^{k \cdot 4}$, $f(x, u) \in \mathbb{R}^{k \cdot 13}$ und sei zu dem $f \in C^{\infty}$ Dann gibt es für die ODE $\dot x = f(x, u)$, weg der Lipschitzstetigkeit, nach Poincare eine eindeutige Lösung $\phi(x, u) \in \mathbb{R}^{k \cdot 13}$ mit 
\begin{align*}
 	\dot \phi(x, u) = f(\phi(x, u), u)
\end{align*}
Man betrachte nun die Differenzialgleichung zum Zeitpunkt $k$, dann folgt für die Differenzierung nach der $x_i$ Komponente
\begin{align*}
	\frac{d}{dx^k_i} \frac{d}{dt^k} \phi(x^k, u^k) = \frac{d}{dx^k_i} f(\phi(x^k, u^k), u^k) \\
\end{align*}
Da $f \in C^{\infty}$ folgt das die Gleichung Sensitivität ist und die Ableitungen vertauscht werden können.
\begin{align*}
\frac{d}{dt^k} \frac{d}{dx^k_i} \phi(x^k, u^k) = \frac{d}{dx^k_i} f(\phi(x^k, u^k), u^k)  &= \frac{d}{dx^k} \left[f(\phi(x^k, u^k), u^k) \right] \cdot \frac{d}{dx^k_i}\phi(x^k, u^k) \\
&+ \frac{d}{du^k}\left[ f(\phi(x^k, u^k), u^k)  \right] \cdot \underbrace{\frac{du^k}{dx^k_i}}_{=0}
\end{align*}
mit $M^k = \frac{d}{dx^k_i} \phi(x^k, u^k) \in \mathbb{R}^{13 \times 13}, i = 1..13$ folgt die ODE  
\begin{align*}
\frac{d}{dt^k} M^k &= \underbrace{\frac{d}{dx^k} \left[f(\phi(x^k, u^k), u^k) \right]}_{\in \mathbb{R}^{13 \times 13}} \cdot M^k
\end{align*}
mit der Anfangsbedingung $M_0^k = I \in \mathbb{R}^{13 \times 13} $ \\\\
Analog folgt für die Differenzierung nach der $u_i$ Komponente
\begin{align*}
\frac{d}{dt^k} \frac{d}{du^k_i} \phi(x^k, u^k) &= \frac{d}{dx^k} \left[f(\phi(x^k, u^k), u^k) \right] \cdot \frac{d}{du^k_i}\phi(x^k, u^k) \\
&+ \frac{d}{du^k}\left[ f(\phi(x^k, u^k), u^k)  \right] \cdot \frac{du^k}{du^k_i}
\end{align*}
mit $N^k = \frac{d}{du^k_i} \phi(x^k, u^k) \in \mathbb{R}^{13 \times 4}, i = 1..4$ folgt die ODE 
\begin{align*}
\frac{d}{dt^k} N^k &= \underbrace{\frac{d}{dx^k} \left[f(\phi(x^k, u^k), u^k) \right]}_{\in \mathbb{R}^{13 \times 13}} \cdot N^k \\
&+ \underbrace{
	\frac{d}{du^k}
		\left[ f(\phi(x^k, u^k), u^k)  \right]
		   }_
{\in \mathbb{R}^{13 \times 4}}
		   \cdot \underbrace{I^k}_{\in \mathbb{R}^{4 \times 4}}
\end{align*}
mit folgender Anfangsbedingung $N_0^k = 0 \in \mathbb{R}^{13 \times 4}$

\subsection*{Jakobimatrix der DAE}

Betrachtet wird nun die DAE 

\[ B (\phi) \cdot \dot{\phi} =
	\begin{pmatrix}
		f(\phi, u) \\ 0
	\end{pmatrix} \]
	
	mit der Matrix \( B =
		\begin{pmatrix}
			1 &   &        &   &      &      &      &      &   &      \\
			  & 1 &        &   &      &      &      &      &   &      \\
				&   & \ddots &   &      &      &      &      &   &      \\
				&   &  &   &      &      &      &      &   &   1   \\
			0	& 0 & 0      & 2q_1 & 2q_2 & 2q_3 & 2q_4 & 0 & \dots & 0
		\end{pmatrix}\)
\\

Dann folgt mit den in Abschnitt "`Jakobimatrix"' genannten Voraussetzungen: \\
Für \(M\):

\begin{align*}
	\frac{\textup{d}}{\textup{d}x_{i-1}} \left( B(\phi) \cdot \dot{\phi} \right) = \frac{\textup{d}}{\textup{d}x_{i-1}} \begin{pmatrix} f \\ 0 \end{pmatrix}
\end{align*}

Auf der rechten Seite ändert sich nichts (die Null in der 14. Zeile ist auch abgeleitet Null), ebenso in den ersten 13 Zeilen auf der linken Seite (weil \(B\) dort eine Einheitsmatrix ist). Neu dazu kommt die Ableitung der letzten Zeile. Diese lautet mit der Kettenregel wie folgt:

\begin{align*}
	& \frac{\textup{d}}{\textup{d}x_{i-1}} \left( \begin{pmatrix} 0 & 0 & 0 & 2q_1 & \dots & 2q_4 & 0 & \dots & 0 \end{pmatrix} \cdot \dot{\phi}\right)  \\
	& = \frac{\textup{d}}{\textup{d}x_{i-1}} \left( \sum_{k=1}^4{2q_k \cdot \dot{q}_k}\right) \\
	& = \frac{\textup{d}}{\textup{d}x} \left( \sum_{k=1}^4{2q_k \cdot \dot{q}_k}\right) \cdot \underbrace{\frac{\textup{d}x}{\textup{d}x_{i-1}}}_{=M} \\
	& = \begin{pmatrix} 0 & 0 & 0 & 2\dot{q}_1 & \dots & 2\dot{q}_4 & 0 & \dots & 0 \end{pmatrix} \cdot M
\end{align*}

Analog gilt für \(N\): Die ersten 13 Zeilen der Linken Seite ändern sich nicht. Auch die rechte Zeile bleibt (bis auf die dazukommende Nullzeile) gleich. Für die letzte Zeile der linken Seite gilt wieder:

\begin{align*}
	& \frac{\textup{d}}{\textup{d}u_{i-1}} \left( \begin{pmatrix} 0 & 0 & 0 & 2q_1 & \dots & 2q_4 & 0 & \dots & 0 \end{pmatrix} \cdot \dot{\phi}\right)  \\
	& = \frac{\textup{d}}{\textup{d}u_{i-1}} \left( \sum_{k=1}^4{2q_k \cdot \dot{q}_k}\right) \\
	& = \frac{\textup{d}}{\textup{d}x} \left( \sum_{k=1}^4{2q_k \cdot \dot{q}_k}\right) \cdot \underbrace{\frac{\textup{d}x}{\textup{d}u_{i-1}}}_{=N} \\
	& = \begin{pmatrix} 0 & 0 & 0 & 2\dot{q}_1 & \dots & 2\dot{q}_4 & 0 & \dots & 0 \end{pmatrix} \cdot N
\end{align*}

Die Ableitung nach \(u\) kommt sowohl für M als auch für N nicht zum tragen, da \(B\) nicht von \(u\) abhängig ist.

Das zu lösende Differentialgleichungssystem sieht nun (in Matrixschreibweise) also folgendermaßen aus:

\[
	 \begin{bmatrix} \dot{x} \\
									 \sum\limits_{k=1}^4{2\cdot q_k \cdot \dot{q_k}} \\
									 \dot{M} \\
									 \begin{pmatrix} 0 & 0 & 0 & 2\dot{q}_1 & \dots & 2\dot{q}_4 & 0 & \dots & 0 \end{pmatrix} \cdot M \\
									 \dot{N} \\
									 \begin{pmatrix} 0 & 0 & 0 & 2\dot{q}_1 & \dots & 2\dot{q}_4 & 0 & \dots & 0 \end{pmatrix} \cdot N
		\end{bmatrix}	
 = \begin{bmatrix} f \vspace{0.6ex} \\
									 0 \vspace{0.6ex} \\
									 \frac{\textup{d}}{\textup{d}x} f \cdot M \vspace{0.6ex} \\
									 0 \vspace{0.6ex} \\
									 \frac{\textup{d}}{\textup{d}x} f \cdot N + \frac{\textup{d}}{\textup{d}u} f \vspace{0.6ex} \\
									 0
	 \end{bmatrix}
\]

bzw.

\[
	 \begin{bmatrix} \dot{x} - f \vspace{.2ex} \\
									 \sum\limits_{k=1}^4{2\cdot q_k \cdot \dot{q_k}} \vspace{.2ex} \\
									 \dot{M} - \frac{\textup{d}}{\textup{d}x} f \cdot M \vspace{.2ex} \\
									 \begin{pmatrix} 0 & 0 & 0 & 2\dot{q}_1 & \dots & 2\dot{q}_4 & 0 & \dots & 0 \end{pmatrix} \cdot M \vspace{.2ex} \\
									 \dot{N} - \frac{\textup{d}}{\textup{d}x} f \cdot N - \frac{\textup{d}}{\textup{d}u} f \vspace{.2ex} \\
									 \begin{pmatrix} 0 & 0 & 0 & 2\dot{q}_1 & \dots & 2\dot{q}_4 & 0 & \dots & 0 \end{pmatrix} \cdot N
	 \end{bmatrix}
		= \textbf{0}
\]

Die Anfangsbedingungen sind weiterhin \(M_0 = \mathbb{I}^{13\times13}\) und \(N_0 = \textbf{0}\). \\
Werden (wegen DAE-Löser) nun auch Anfangsbedingungen für \(\dot{M}\) und \(\dot{N}\) gebraucht???


\subsection*{Hessematrix}

\begin{align*}
\frac{d}{dt^k} \frac{d}{dx^k_j} \frac{d}{dx^k_i} \phi(x^k, u^k) &= \frac{d}{dx^k_j} \left\{ \frac{d}{dx^k} \left[f(\phi(x^k, u^k), u^k) \right] \cdot \frac{d}{dx^k_i}\phi(x^k, u^k) \right\}\\
&=\left\{\frac{d}{dx^k} \frac{d}{dx^k} \left[f(\phi(x^k, u^k), u^k) \right] \cdot \frac{d}{dx^k_i}\phi(x^k, u^k) + \frac{d}{du^k} \frac{d}{dx^k} \left[ f(\phi(x^k, u^k), u^k)  \right] \cdot \underbrace{\frac{du^k}{dx^k_i}}_{=0} \right\} \\
&+ \frac{d}{dx^k} \left[f(\phi(x^k, u^k), u^k) \right] \cdot \frac{d}{dx^k_i} \frac{d}{dx^k_i}\phi(x^k, u^k)
\end{align*}
mit $M^k = \frac{d}{dx^k_i} \phi(x^k, u^k) \in \mathbb{R}^{13 \times 13}, i = 1..13$ und $O^k = \frac{d}{dx^k_i} \frac{d}{dx^k_j} \phi(x^k, u^k) \in \mathbb{R}^{13 \times 13 \times 13}, i = 1..13, j=1..13$  folgt die ODE 
\begin{align*}
\frac{d}{dt^k} O_k &=\frac{d}{dx^k} \frac{d}{dx^k} \left[f(\phi(x^k, u^k), u^k) \right] \cdot M_k + \frac{d}{dx^k} \left[f(\phi(x^k, u^k), u^k) \right] \cdot O_k
\end{align*}
mit folgender Anfangsbedingung $H_0^k = 0 \in \mathbb{R}^{13 \times 13 \times 13}$
\begin{align*}
\frac{d}{dt^k} \frac{d}{du^k_j} \frac{d}{dx^k_i} \phi(x^k, u^k) &= \frac{d}{du^k_j} \left\{ \frac{d}{dx^k} \left[f(\phi(x^k, u^k), u^k) \right] \cdot \frac{d}{dx^k_i}\phi(x^k, u^k) \right\}\\
&= \left\{ \frac{d}{dx^k} \frac{d}{dx^k} \left[f(\phi(x^k, u^k), u^k) \right] \cdot \frac{d}{du^k_i}\phi(x^k, u^k) \right. \\
& \left. + \frac{d}{du^k} \frac{d}{dx^k} \left[f(\phi(x^k, u^k), u^k) \right] \cdot \frac{du^k}{du_j^k} \right\} \cdot \frac{d}{dx^k_i}\phi(x^k, u^k) \\
&+ \frac{d}{dx^k} \left[f(\phi(x^k, u^k), u^k)\right] \cdot \frac{d}{du^k_j} \frac{d}{dx^k_i}\phi(x^k, u^k) 
\end{align*}
mit $M^k = \frac{d}{dx^k_i} \phi(x^k, u^k) \in \mathbb{R}^{13 \times 13}, i = 1..13$, $N^k = \frac{d}{du^k_j} \phi(x^k, u^k) \in \mathbb{R}^{13 \times 4}, i = 1..13$ und $P^k = \frac{d}{du^k_j} \frac{d}{dx^k_i} \phi(x^k, u^k) \in \mathbb{R}^{13 \times 13 \times 4}, i = 1..13, j = 1..4$  folgt die ODE in der Matrixschreibweise
\begin{align*}
P^k = H_{x, u} \phi(x, u) = J_{u} {(M_k)^T} &= \left\{\nabla_u \left(\nabla_x \phi \cdot \nabla_x f(\phi, u)\right)\right\}^T \\
&= \left\{ \nabla_x \phi \cdot \nabla_u \nabla_x f(\phi, u) + \nabla_{ux} \phi \cdot  \nabla_x f(\phi, u)  \right\}^T \\
&= \left\{ J_u \nabla_x f(\phi, u) \cdot J_x \phi + \nabla_{ux} \phi \cdot  \nabla_x f(\phi, u)  \right\} \\
&= \left\{ \left(J_x \nabla_x f(\phi, u) \cdot J_u \phi + J_u \nabla_x f(\phi, u) \right)   \cdot J_x \phi +   J_x f(\phi, u) \cdot J_u \nabla_x \phi  \right\} \\
&= \left\{ \underbrace{\left(H_{x, x} f(\phi, u) \cdot N^k \phi + H_{u, x} f(\phi, u) \right)   \cdot M^k }_{\text{Hier ein Dimesionsproblem, Warum?}}+  J_x f(\phi, u) \cdot  P^k  \right\} 
\end{align*}

\end{document}