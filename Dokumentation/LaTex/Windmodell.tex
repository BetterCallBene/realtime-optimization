\documentclass{scrartcl}

\usepackage{enumerate, amsmath, amssymb,amsthm, amstext}
\usepackage[english,ngerman]{babel}
\usepackage[latin1]{inputenc}   

\begin{document}

\parindent=0pt

\section{Modellierung Wind}

\subsection{Kraft:}

\textbf{Winddruck:} 
\[ P_{Wind} = c_P \cdot \frac{\rho_{Luft}}{2} \cdot v^2 \]

\textbf{Windkraft:}
\[ F_{Wind}= A \cdot P_{Wind} \]

\begin{tabular}{ccc}
	\(c_P\) & Druckbeiwert & z.B. \(c_P=0.7-2\) \\
	\(\rho_{Luft}\) & Luftdichte in \(\frac{kg}{m^3}\) & abh�ngig von H�he und Temperatur (ca.\(1.0 - 1.3 \frac{kg}{m^3})\)
\end{tabular}

\vspace{2em}

\textbf{Windst�rke:}

\begin{tabular}{ccccl}
	Beaufort & m/s & km/h & Bezeichnung & Wirkung \\
	0  & 0.0-0.3 			& 0-1 					& Flaute &\\
	1  & 0.3-1.6			& 1-5 					& leiser Zug &\\
	2  & 1.6-3.4			& 6-11					& leichte Brise &\\
	3  & 3.4-5.5			& 12-19					& schwache Brise &\\
	4  & 5.5-8.0			& 20-28					& m��ige Brise & Zweige bewegt, Papier von Boden gehoben\\
	5  & 8.0-10.8			& 29-38					&	frische Brise &\\
	6  & 10.8-13.9		&	39-49					& starker Wind &\\
	7  & 13.9-17.2		& 50-61					& steifer Wind & B�ume schwanken, Widerstand beim Gehen\\
	8  & 17.2-20.8		& 62-74					& st�rmischer Wind & gro�e B�ume bewegt, Zweige abgebrochen\\
	9  & 20.8-24.5		& 75-88					& Sturm \\
	10 & 24.5-28.5		& 89-102				& schwerer Sturm \\
	11 & 28.5-32.7		& 103-117				& orkanartiger Sturm \\
	12 & \(\geq\)32.7	& \(\geq\) 117	& Orkan \\
\end{tabular}

\vspace{2em}

F�r uns k�nnte interessant sein:
\begin{enumerate}
	\item Windst�rke 7-8 \(\rightarrow\) Gegenwind bei Skifahrer mit ca. \(60 \frac{km}{h}\)
	\item h�chste m�gliche Windst�rke?
	\item Zeitverlust der Rechnung?
\end{enumerate}

\subsection{Drehmoment:}

\textbf{Betrag:}
\[ M=r\cdot F \]
\[ M = \sum_i{r_i \cdot F_i} \]

\end{document}