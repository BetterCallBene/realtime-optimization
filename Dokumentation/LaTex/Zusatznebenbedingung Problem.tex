\documentclass[12pt,a4paper]{article}
\usepackage[utf8]{inputenc}
\usepackage[german]{babel}
\usepackage[T1]{fontenc}
\usepackage{amsmath}
\usepackage{amsfonts}
\usepackage{amssymb}
\usepackage[left=2cm,right=2cm,top=2cm,bottom=2cm]{geometry}

\setcounter{MaxMatrixCols}{20}


\begin{document}


  
  \begin{align*}
  L^{k}(y) = \sum_{i=k}^{N-1} f_{i}(s_{i},q_{i})
  + \lambda_{k}^{T}(x_{k} - s_{k})
  + \sum_{i=k}^{N-1} \lambda_{i+1}^{T} (h_i (s_i ,q_i ) - s_{i+1})
  + \mu_{i+1}(p_i(s_i)-1)
  \end{align*}
  Betrachte das $\mu $als letzte Kompnente von $\lambda$. Dadurch verändert sich die Form des gewählten $y$ nicht und die Bedingung bleibt weiterhin.
  \begin{align*}
  \nabla_{y} L^{k}(y)  = 0
  \end{align*}
  und das exakt Newton-Raphson-Verfahren
  \begin{align*}
  y_{i+1} = y_i + \Delta y_i
  \end{align*}
  bei dem jedes $ \Delta y_i $ die Lösung des linearen Systems
  \begin{align*} 
  \nabla_{y} L^{k}(y_{i}) + \nabla_{y}^{2} L^{k}(y_{i}) \Delta y_{i} = 0
  \end{align*}
  ist.
  
  $\nabla_{y}^{2} L^{k}(y)$ hat dann folgende Form.\\
  \begin{align*} 
  \nabla_{y}^{2} L^{k}(y) =
  \begin{pmatrix}
    & -E  &     &     &     &     &     &     &     &     &  & & &  \\
-E  & Q_k & M_k & A_k^{T} & C_k^T &    &     &     &     &   &  &  & &   \\
    & M_k^{T} & R_k & B_k^{T} &   &    &    &    &    &   &  & & &  \\
    & A_k & B_k &   &  & -E  &     &     &     &     &      & & &   \\
    & C_k &  &     &     &     &   &   &  &  & & & &  \\
    &  &  & -E  &  & Q_{k+1} & M_{k+1} & A_{k+1}^{T} & C_{k+1}^T &  & & & &  \\
    &  &  &     &  & M_{k+1}^{T} & R_{k+1} & B_{k+1}^{T} &  &  & & & &  \\
    &  &  &     &  & A_{k+1} & B_{k+1} &    &    &     &  &   & &    \\
    &  &  &     &  & C_{k+1} &    & &  & \ddots &     &     &  &   \\
    &  &  &   &  & & & & \ddots & Q_{N-1} & M_{N-1} & A_{N-1}^{T} & C_{N-1}^{T} &  \\
    &  &  &   &  & & &    & & M_{N-1}^{T} & R_{N-1} & B_{N-1}^{T} & & \\
    &  &  &   &  & & &    & & A_{N-1}     & B_{N-1} &  &  & -E \\
    &  &  &   &  & & &    & & C_{N-1}     &     &    &  &      \\
    &  &  &   &  & & &    & &             &     & -E & & Q_N
\end{pmatrix}
  \end{align*}
  \\
  Mit $ A_i := \dfrac{\partial h_i}{\partial s_i} $, 
  $ B_i := \dfrac{\partial h_i}{\partial q_i} $,
  $
  \begin{pmatrix}
  Q_i & M_i \\
  M_i^{T} & R_i
  \end{pmatrix} := \nabla_{s_i,q_i}^{2}L^{k} $ und 
  $ Q_N := \nabla_{s_N}^{2}L^{k} $ .\\
  $ C_i := \dfrac{\partial p_i}{\partial s_i} $
  
  

\end{document}