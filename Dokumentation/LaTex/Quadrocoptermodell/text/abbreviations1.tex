\chapter{Abkürzungen} \label{abkuerzungen}

\section{Abk\"urzungen} \label{sec:abkuerz}
\begin{longtable}[l]{p{0.2\tw} p{0.5\tw} p{0.1\tw}}
EIS     & Elektrochemische Impedanzspektroskopie                             \\
KKT	  & Kramer-Kronig-Transformation						     \\
\end{longtable}

\section{Formelzeichen} \label{sec:formulaSym}
\begin{longtable}[l]{lll}
$I$             & Strom                          & $A$ \\
\end{longtable}

%\section{Definitions}

%\begin{longtable}[l]{ll}
%sinkage   & distance of the wheel that sinks into the soft soil \\
%rut depth & the rut that remains after deformation from the wheel
%\end{longtable}

%\section{Coordinate system} \label{sec:coosys}

%The LSS shall use the right-hand rover vehicle $X_{RB}Y_{RB}Z_{RB}$ coordinate system defined as
%follows and shown in figure~\ref{fig:coordinate_system}:
%
%\begin{itemize}
%\setlength{\parskip}{1pt}
%\item Origin at the upper surface of the SES base-plate with the Rover in its stowed configuration
%\item $X_{RB}$-axis in the fore-aft direction of Rover motion, positive forwards (roll axis)
%\item $Y_{RB}$-axis to form right hand set, positive towards left side (pitch axis)
%\item $Z_{RB}$-axis parallel to the gravity vector in nominal attitude on a flat horizontal plane,
%normal to Rover base, positive upwards (yaw axis)
%\end{itemize}

%\begin{figure}[htb]
%\centering
%\includegraphics[height=7cm]{coo}
%\caption{Rover coordinate system, from \cite{analysSpec}} \label{fig:coordinate_system}
%\end{figure}
