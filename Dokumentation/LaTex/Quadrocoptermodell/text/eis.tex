\chapter{Die \eis}\label{eis}
Wir wollen nun die in Kapitel \ref{sys} verinnerlichten mathematischen Methoden auf elektrochemische Systeme übertragen. Auf diesem Gebiet wurde von Dambrowski \cite{Dambrowski2013} eine elementare Arbeit geleistet, die wir in diesem Kapitel vorstellen werden. Im vorherigen Kapitel wurde das Verfahren der elektrochemischen Impedanzspektroskopie schon kurz vorgestellt. Dies wird nun intensiviert und wir lernen einen Ansatz zur Validierung der durch die \eis entstandenen Messdaten.  \\
In \ref{sys:eis} wurde die Impedanz vorgestellt: Wir wollen sie konkret definieren.

\begin{defi}\label{eis:impedanz} (Impedanz)
	Sei $i(t)$ der Eingabestrom und $u(t)$ die Spannungsantwort. Dann ist die \textit{Impedanz} von einem reellen Faltungssystem eine komplexwertige Funktion $Z: I \rightarrow \mathbb{C}$ und es gilt:
	\begin{align}
		Z(\omega):= H(\omega) = \frac{\ft{u(t)}(\omega)}{\ft{i(t)}(\omega)}	= \frac{U(\omega)}{I(\omega)}
	\end{align} 
\end{defi}
\begin{bem}
	Dabei wird in der Praxis meist die Batterie durch einen sinusförmigen Strom angeregt und man bekommt eine um einen Winkel verschobene Spannungsantwort.
\end{bem}
\begin{bem}
	Im Fall eines elektrochemischen Systems $\mathscr{S}$ ist die Impedanz der Frequenzgang. Wenn die Impedanz bekannt ist, kann man für jede Systemanregung die Antwort im Voraus berechnen. 
\end{bem}
                                                                                                                                                      Diese Definition reicht nicht aus, um sicherzustellen, dass die Impedanz $Z(\omega)$ existiert, da die Existenz von der Fourier - Transformation nur implizit angenommen wird. Auch die LTI - Eigenschaft alleine garantiert die Existenz nicht. \\
                                                                                                                                                      
Die Impedanz kann nun folgendermaßen sichergestellt werden:
                                                                                                                                                                                                                                                                                                            \begin{satz}\label{eis:ex1}(Existenz der Impedanz I)                                                                                                                                                    Die Existenz der Impedanz $Z(\omega)$ eines elektrochemischen System $\mathscr{S}$ wird garantiert, wenn es ein reelles LTI - System, dessen Impulsantwort $h(t) \in \lint{1}{\mathbb{R}}$, ist.                                                                                                                                                                                                                                                                                                              \begin{proof}                                                                                                                                                       Nach Definition der Impedanz muss das Faltungssystem reell sein. Nach Satz \ref{sys:faltung} folgt sofort die Behauptung, wenn das System $\mathscr{S}$ LTI und $h(t) \in \lint{1}{\mathbb{R}}$ ist.                                                                                                                                                \end{proof}                                                                                                                                                                                                                                                                                                            \end{satz}
                                                                                                                                                                                                                                                                                                           \begin{satz}\label{eis:ex2}(Existenz der Impedanz II)
                                                                                                                                                    Für alle reellen BIBO - stabilen Faltungssysteme ist die Existenz für die Impedanz gesichert.                                                                                                                                                       \begin{proof}
Da per Definition $\mathscr{S}$ reell ist und nach Satz \ref{sys:stabil} sofort folgt, dass die Impulsantwort in $h(t) \in \lint{1}{\mathbb{R}}$ liegt. Folgt mit letztem Satz \ref{eis:ex1}  die Behauptung.
\end{proof}                                                                                                                                                     \end{satz}

\begin{bem}
In verschiedenen praktikablen Situationen sind diese Konditionen verwendbar. Zu bemerken ist, dass dies nur hinreichende, aber nicht notwendige Bedingungen für die Existenz der Impedanz darstellen. 
\end{bem}
                                                                                                                                                                                                                                                                                                    \begin{satz}\label{eis:be}                                                                                                                                                                                                                                                                                       (Kausalität und Beschränktheit) 
                                                                                                                                                                                                                                                                                                   Ein kausaler LTI Operator $T: \mathscr{X} \rightarrow \mathscr{N}$ ist beschränkt und damit definiert er ein Faltungssystem im Sinne von Satz \ref{sys:faltung}.
                                                                                                                                                                                                                                                                                                   \begin{proof}
                                                                                                                                                                                                                                                                                                   Dies ist ein tiefliegender Satz und wird in \cite{Dambrowski2013} bewiesen.
                                                                                                                                                                                                                                                                                                  \end{proof}
                                                                                                                                                                                                                                                                                                  Daher kann die wichtige Systemvoraussetzung, für die Existenz von Impedanz $Z(\omega)$, \textit{Beschränktheit} (in Satz \ref{sys:faltung}) durch die Kausalität ersetzt werden. 
                                                                                                                                                                                                                                                                                                      \end{satz}
                                                                                                                                                                                                                                                                                                     \begin{satz}\label{eis:ex3}(Existenz der Impedanz III)
                                                                                                                                                                                                                                                                                                    Die Existenz der Impedanz $Z(\omega)$ eines elektrochemischen Systems $\mathscr{S}$ ist garantiert, wenn es reell, LTI und kausal ist. 
                                                                                                                                                                                                                                                                                                    \begin{proof}
                                                                                                                                                                                                                                                                                                    Mit Lemma \ref{eis:be} und Satz \ref{sys:faltung}.
                                                                                                                                                                                                                                                                                                 
\end{proof}                                                                                                                                                                                                                                                                                                                                                                                                                                                                                                                                                                                                                                                                                                                                                                                                                                                                                                         
                                                                                                                                                                                                                                                                                                      \end{satz}
                                                                                                                                                                                                                                                                                                  
                                                                                                                                                                                                                                                                                                    
In den kennengelernten Existenzsätzen ist die Linearität und Zeitinvarianz das Hauptkriterium.                                                                                                                                                                                                                                                                                                                                                                                                                                                                                                                                                                                                               Da aber die Batterien weder linear noch zeitinvariant sind, ist eine Linearisierung durch kleine Spannungen als Anregung möglich. In der Praxis findet die Anregung der Batterie aber nicht mit der Spannung statt, sondern mit dem Strom. Dabei muss darauf geachtet werden, dass die Spannungsantwort im Bereich von $\mp 10 mV$ liegt\cite{Huet1998}.\\
                                                                                                                                                                                                                                                                                                     Stellt man nun eine feste Vorgeschichte des elektrochemischen Systems her, z.B. durch erneutes Anfahren eines bestimmten Ladezustandes, lässt sich dadurch eine - zumindest temporäre - Zeitinvarianz herstellen. Nach Herstellung sind außerdem noch batteriespezifische Wartezeiten einzuhalten. Reagiert die Batterie auf ein beschränktes Anregungssignal beschränkt, so ist mit Satz \ref{eis:ex2} die Existenz der Impedanz gegeben. \\\\
                                                                                                                                                                                                                                                                                                                                                                                                                                                                                                                                                                                                                                                                                                                                                                                                                                                                                                                                                                                                                                                                                                                                                                                                                  \section{Die elektrochemische Impedanspektroskopie und die Hilbert - Transformation}
                                                                                                                                                                                                                                                                                                                                                                                                                                                                                                                                                                                                                                                                                                                                                                                                                                                                                                                                                                                                                                                                                                                                                                                                                  Die vorher beschriebene Methode ist nur experimenteller Natur. Ein exaktes und beliebtes Mittel zur Validerung der Daten, ist die Hilbert - Transformation und ihre Spezialfälle.   \\\\
                                                                                                                                                                                                                                                                                                                                                                                                                                                                                                                                                                                                           Unglücklicherweise wird für den Satz von Titchmarsh (Satz \ref{ht:titch}) vorausgesetzt, dass das System $\mathscr{S}$ ein Faltungssystem ist. Diese Annahme ist aktuell nötig und wird in Satz \ref{sys:kausalität} angewendet.	Daher gilt, wenn $\mathscr{S}$ ein reelles Faltungssystem ist und wenn die Hilbert - Transformation fehlschlägt, das System $\mathscr{S}$ nicht kausal ist oder unendliche Energie besitzt. Weiß man hingegen, über die Faltungseigenschaften des elektrochemischen Systems nicht Bescheid so folgt für
                                                                                                                                                                                                                                                                                                      \begin{itemize}
                                                                                                                                                                                                                                                                                                     \item und sind $\Re{H(\omega)}, \Im{H(\omega)}$ Hilbert - Paare so kann man nichts folgern.
                                                                                                                                                                                                                                                                                                     \item sind $\Re{H(\omega)}, \Im{H(\omega)}$ keine Hilbert - Paare, so repräsentieren die Messdaten keine richtige Impedanzen nach \ref{eis:impedanz}
\end{itemize}                                                                                                                                                                                                                                                                                                        

Für die Herleitung der Hilbert - Beziehungen muss das System weder linear noch zeitinvariant sein. Ein konkretes nicht lineares Beispiel, welches die Hilbert - Transformation erfüllt, wird in \cite{Dotz2013a} konkret angegeben. \\\\

\begin{figure}[ht]
	\centering
	\def\svgwidth{0.9\columnwidth}
	\input{images/impedanz.pdf_tex}
	\caption{Veranschaulichung der Existenzsätze der Impedanz}
	\label{fig:impedanz}
\end{figure}
%
%
%Auch mit Hilfe der Passivität ist die Existenz der Impedanz geben.
%
%\begin{satz}(Existenz der Impendanz IV)
%Sei das System $\mathscr{S}$ LTI und passiv  q $\Re{H(\omega}) \geq 0$ für fast alle $\omega$'s dann folgt, dass die Messdaten der EIS richtig.
%\begin{proof}
%Sei per Definition $\mathscr{S}$, LTI und passiv dann folgt, nach den Satz \ref{sys:passiv}, dass das System kausal ist. Aber wenn, dass System $\mathscr{S}$ LTI und kausal ist folgt nach \ref{eis:be} \ref{sys:faltung}, dass das System ein Faltungssystem ist. 
% 
%\end{proof}
%\end{satz}
 

                                                                                                                                                                                                                                                                                                                                                                                                                                              
                                                                                                                                                       
                                                                                                                                                                                                                                                                                                         
                                                                                                                                                
                                                                                                                                                      
                                                                                                                                         
                                                                                                                                            