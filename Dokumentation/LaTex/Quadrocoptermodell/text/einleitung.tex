\chapter{Einleitung}
In Deutschland wurde vor allem in den 80er Jahren des letzten Jahrhunderts vehement der Ausstieg von der Kernkraftenergie gefordert und nach Alternativen gesucht. Jetzt – 30 Jahre später – gibt es immer noch den sogenannten „Atomstrom“, aber wir sind einer Energiewende deutlich näher als damals. Ein endgültiger atomarer Ausstieg ist von Seiten der Politik bis 2022 geplant. Bis dahin muss jedoch noch in dem Bereich der erneuerbaren Energien viel Arbeit geleistet werden. \\\\
\begin{figure}[ht]
	\centering
  	\includegraphics[width=0.75\columnwidth]{co2.pdf}
  	\caption{Vergleich zwischen konventionellen Systemen, PV und PV - Batteriesystem im Zug auf Treibhausgasemissionen \cite{Sievers2013}}
  	\label{fig:co2}
\end{figure}
Einen Anfang war, als vor mehr als zehn Jahren Photovoltaikanlagen (PV) in Privathaushalten immer beliebter wurden und seitdem 15 Millionen Hausdächer eine Versorgungsleistung im Bereich der Solarenergie stellen. Diese großen Erfolge sind darauf zurückzuführen, da es zunächst eine staatliche Einspeisevergütung auf gesetzlicher Grundlage gab. Diese Einspeisevergütung ist jedoch in den letzten Jahren deutlich gesunken. Mittlerweile ist der Bezug von Strom teurer als der Verkauf. Deswegen ist es sinnvoller den produzierten Solarstrom als Eigenverbrauch zu nutzen, anstatt ihn zu verkaufen. Dies bietet einen besonderen Anreiz für ein Batteriesystem. Allerdings ist ein Batteriesystem mit hohen Investitionskosten für den Konsumenten verbunden, es kann trotzdem von einer guten ökonomischen Perspektive ausgegangen werden, in dem eine erhebliche Steigerung des Eigenverbrauchs erreicht wird. Photovoltaik und Photovoltaik-Batteriesysteme erreichen mit der Herstellung und Alles in Allem deutlich geringere Treibhausemissionen als konventionelle Systeme (Abb. \ref{fig:co2}) \cite{Sievers2013}.\\\\
Es stellt sich nun die Frage, ob ein Batteriesystem als Zukunftsmodell fungieren kann. Eine Besserung des Batteriesystems und Forschung im Bereich der Klassifizierung von Batterien bringt uns dieser Zukunftsvision deutlich näher. Allerdings muss auch der finanzielle Aspekt berücksichtigt werden, denn Batterien sind teuer und haben eine beschränkte Laufzeit ab Erzeugung, gleich wenn dies in den letzten zehn Jahren verbessert wurde.

 

