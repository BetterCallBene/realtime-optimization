\chapter{Zusammenfassung und Ausblick}
Durch diese Arbeit soll die Problemstellungen anhand der Grundlage der Systemtheorie, Einführung in die EIS und der Modellbildung eines elektrochemischen Systems, sowie das numerische Lösen durch verschiedene Verfahren der Hilbert-Transformation näher gebracht werden. Es wurde dem Begriff System eine für uns adäquate Definition gegeben und daran gewisse Eigenschaften charakterisiert.\\
Dabei wurden zwei Arten von linearen Systemen vorgestellt, ihr Zusammenhang hervorgehoben und am Beispiel der elektrochemischen Impedanzspektroskopie veranschaulicht. Ein besonderes Augenmerk wurde auf die Kausalität gerichtet, die mit Hilfe der Hilbert-Transformation – unter gewissen Voraussetzungen – bestimmt werden kann. Jene spielt eine wichtige Rolle, um die Existenz der Impedanz zu ermitteln.
Nach den Existenzsätzen der Impedanz wurden verschiedenste Darstellungsformen der Hilbert-Transformation aufgezeigt. Im Besonderen das artverwandte Fourierintegral, die logarithmische Darstellung der KKT und das Aufzeigen der HT auf Basis des Faltungsintegrals
Folgend wurde auf die Grundlagen der Modellbildung für elektrochemische Speicher eingegangen. Die Komponenten eines Batteriemodells wurden vorgestellt, im Besonderen das Warburg - Element. Die Beweisführung ergab, dass die Approximation des Warburg-Elements durch die RC - Glieder existiert und den Satz von Titchmarsh erfüllt.  
Vier verschiedene Verfahren zur numerischen Berechnung der HT konnten vorgestellt und klassifiziert werden. Hervorzuheben ist, dass das Z-Hit-Verfahren aus der logarithmischen KKT ermittelt und bewiesen wurde.
\subsection{Ausblick}
Es muss validiert werden, ob die Passivität eines Systems für die Existenz der Impedanz eine weittragende Rolle spielt. Des Weiteren muss die Erweiterung der Modellbildung für elektrochemische Speicher angedacht werden, im Speziellen für das Constant Phase Element. Da ist zu überprüfen, ob eine ähnliche Approximation mit Hilfe von RC - Gliedern stattfinden kann, um so den Satz von Titchmarsh zu erfüllen. 
Da sich bei den numerischen Verfahren der Z-Hit als Optimum herausgestellt hat, muss geprüft werden, ob er mit Hilfe von elektrischen Bauelementen als Testautomat konstruiert werden kann.