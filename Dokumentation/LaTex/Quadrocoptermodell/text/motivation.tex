\chapter{Motivation}
Zur Bestimmung der komplexen Impedanz $Z(\omega)$ eines elektronischen Speichers kann als Messverfahren die elektrochemische Impendanzspektroskopie (EIS) angewandt werden und wurde in der Elektrochemie immer beliebter \cite[Seite XIII]{MacDonald2005}. Die EIS gibt Hilfestellung zur Modellierung, Charakterisierung und einer Zugangsdiagnostik in Batteriespeichersystemen.  Die Batterie hat ein nicht-lineares Verhalten, das sich nur schwer erfassen lässt \cite[Seite 3]{Mauracher1996}. Das elektrochemische System kann nur zeitlich begrenzt in einen pseudo -linearen Zustand gebracht werden. Um Feststellen zu können, ob eine Batterie sich in einem derartigen pseudo - linearen Zustand befindet, muss ein Prüfmittel herangezogen werden. In der Praxis wird oft die Kramers - Kronig - Transformation als Validierungsmittel der Impedanz verwendet \cite{Boukamp1993} \cite{Urquidi-Macdonald1986} \cite{Triverio2006b}. Der komplexe Widerstand (Impedanz) ist als Funktion in Abhängigkeit von der Frequenz definiert und unter gewissen Voraussetzungen steht der Realteil mit dem Imaginärteil in Beziehung. Dabei muss sie u. a. die Kramers-Kronig-Transformation erfüllen \cite[Seite 7]{MacDonald2005}. \\\\
Die Anwendung der KKT stellt uns allerdings vor einige Probleme:
\begin{itemize}
 	\item Erfüllen die Daten die Voraussetzungen der KKT?
	\item Bei der elektrochemischen Impendanzmessung kann nur eine geringe Anzahl von Datenpunkten ausgewertet werden. Wie findet die Approximation der restlichen notwendigen Daten statt?
	\item Nicht alle Frequenzbereiche, die für die Auswertung der KKT notwendig wären können messtechnisch erfasst werden.
	\item Die Existenz der KKT ist nur über den Cauchy - Hauptwert definiert.
\end{itemize}
Diese Problemstellungen sollen in dieser Arbeit gelöst werden.