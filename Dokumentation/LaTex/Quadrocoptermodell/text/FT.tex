\chapter{Fouriertransformation}\label{ft}
\begin{satz}\label{ft:def:fourier}(Fouriertransformation) Sei $\omega \in \mathbb{R}$, $f \in \lint{1}{\mathbb{R}}$ und man betrachte die Formel
\begin{align}
	\mathscr{F}\{f(t)\}(\omega) = \int_{-\infty}^{\infty}{f(t) \exp{(-i \omega t)}} \mathrm{dt}
\end{align}
\end{satz}\cite[Seite 1163]{Mathe2012}
\begin{satz}(Umkehrformel für Fouriertransformation)\label{ft:umkehr}
Mit $x(t) \in L^2(\mathbb{R})$ ist auch $\tilde{x}:=\mathscr{F} \{x(t)\}(\omega) \in L^2(\mathbb{R})$ und es gilt
\begin{align}
	x(t) = \mathscr{F}^{-1}\{\tilde{x}(\omega)\}(t) = \frac{1}{2 \pi} \cdot \int_{-\infty}^{\infty}{\tilde{x}(\omega) e^{i \omega t}} \mathrm{d\omega} =  \frac{1}{2 \pi} \cdot \int_{-\infty}^{\infty}{\mathscr{F} \{x(t)\}(\omega) e^{i \omega t}} \mathrm{d\omega}
\end{align}
Ferner gilt die \textbf{Formel von Plancherel}
\begin{align}\label{ft:plancherel}
\langle \mathscr{F}x, \mathscr{F}y \rangle_{L^2} = 2 \pi \langle x, y \rangle_{L^2}
\end{align}
mit $x$, $y \in L^2(\mathrm{R})$ 
\end{satz}\cite[Seite 1163]{Mathe2012}
\begin{defi}
Der Funktionenraum 
\begin{align}
	S(\mathbb{R}):= \left\{x \in C^{\infty}(\mathbb{R}) \vert t^p x^{k} \in L(\mathbb{R}) \forall p\text{, } k \in \mathbb{N}_0\right\}
\end{align}
wird Schwartz-Raum oder Menge der schnell abfallenden Funktionen genannt.
\end{defi}
\begin{satz}
Sei $x \in L^2(\mathbb{R})$, $y \in L(\mathbb{R})$ (oder umgekehrt), so gilt 
\begin{align}
	\mathscr{F}\{x * y\}(s) = \mathscr{F}\{x\}(s) \cdot \mathscr{F}\{y\}(s)
\end{align}
Sei $x \in L^2(\mathbb{R})$, $y \in S(\mathbb{R})$ (oder umgekehrt), so gilt
\begin{align}
	\mathscr{F}\{x y\}(s) = (\mathscr{F}\{x\} * \mathscr{F}\{y\})(s)
\end{align}
\end{satz}
\section{Einige Fouriertransformationen}\label{ft:tab}
\begin{center}
\begin{tabular}{|c|c|}
\hline
\rule[-1ex]{0pt}{2.5ex} Signal & Fouriertransformierte \\  
\rule[-1ex]{0pt}{2.5ex} $x(t)$ & $\mathscr{F}\{x(t)\}(i \omega)$ \\
\hline  
\rule[-1ex]{0pt}{2.5ex} $\sgn{t}$ & $\sqrt{\frac{2}{\pi}} \cdot \frac{1}{i \omega} $ \\ 
\hline
\rule[-1ex]{0pt}{2.5ex} $t^n$ & $- i \frac{(-i \omega)^{n-1}}{(n-1)!}\sgn{\omega}$\\ 
\hline 
\end{tabular} 
\end{center} 

