\documentclass[12pt, a4paper]{scrartcl}
\usepackage{fullpage}
\usepackage[utf8]{inputenc} %Zeichensatzkodierung (auch Sonderzeichen -> ä,ö,ü)
\usepackage[english, ngerman]{babel} %Sprache -> Silbentrennung, automatisch generierte Texte auf deutsch; ngerman -> NEUE deutsche Rechtschreibung
\usepackage[T1]{fontenc} %Sonderzeichen allg
\usepackage{lmodern} %PDF-optimierte Schrift

\usepackage{amsmath, amssymb, amsbsy, amsthm} %Mathepakete ; amsthm: Theorem-Umgebung; muss nach amsmath eingebunden werden
\usepackage{graphicx} % Grafiken einfügen
%\usepackage{listings} % zum Einbinden von Code, ist bereits in "`mcode"' enthalten
\usepackage{tabularx}
%\usepackage{extarrows} % Pfeile mit Beschriftung, deren Länge automatisch angepasst wird

\usepackage{setspace} % Zeilenabstand verändern

\usepackage{bibgerm} % Für das Literaturverzeichnis

\setlength{\parindent}{0pt}

\newcommand{\R}{\mathbb{R}}


\begin{document}

%\section{Modelle}\label{Bla}
%
%\subsection{Mathematisches Modell für + Konfiguration}\label{sub:konfg_plus}
%Mit \ref{sec:konfg_matrix} folgt für die + Konfiguration 
%\begin{align}
    %{_{0}\dot{r}_{1}} &= A_{0, 1} {_{1}v_{1}} \\
    %m_{B_1} \frac{d_1}{dt} {_{1}v_{1}} &+ {_{1}\omega_{1}} \times m_{B_1} {_{1}v_{1}} = m_{B_1} A_{1, 0} \cdot \begin{pmatrix} 0 \\ 0 \\ -g \end{pmatrix} + \sum_{i = 1}^4 {\begin{pmatrix} 0 \\ 0 \\ F_i \end{pmatrix}} \\
    %{{\dot{A}_{0, 1}}} &= {A_{0, 1}} \tilde{_{1} {\omega}_{1}}\\
    %{_{1} I_{1}} {_{1} \dot{\omega}_{1}} &+ {_{1} {\omega}_{1}} \times {_{1} I_{1}} {_{1} {\omega}_{1}} + \sum_{i=1}^{4}{({_{1}{\omega}_{1}} \times e_z) \cdot I_{M} {_{M_i}\omega_{M_i}} } \\
    %&= \begin{bmatrix} 0 \\ 0 \\ -\sum_{i = 1}^{4}{M_i} \end{bmatrix} + (\tau_{R} + \tau_{P}) = \begin{bmatrix} d(F_2 - F_4) \\ d(F_3 - F_1) \\ -\sum_{i = 1}^{4}{M_i} \end{bmatrix}
%\end{align}
%
%Mit \ref{gl:MDrag} und \ref{gl:schub}
%\begin{align}
    %{_{0}\dot{r}_{1}} &= A_{0, 1} {_{1}v_{1}} \\
    %m_{B_1} \frac{d_1}{dt} {_{1}v_{1}} &+ {_{1}\omega_{1}} \times m_{B_1} {_{1}v_{1}} = m_{B_1} A_{1, 0} \cdot \begin{pmatrix} 0 \\ 0 \\ -g \end{pmatrix} + {\begin{pmatrix} 0 \\ 0 \\ \sum_{i = 1}^4 k_{T} {_{M_i}\omega^2_{1, M_i}} \end{pmatrix}} \\
    %{{\dot{A}_{0, 1}}} &= {A_{0, 1}} \tilde{_{1} {\omega}_{1}}\\
    %{_{1} I_{1}} {_{1} \dot{\omega}_{1}} &+ {_{1} {\omega}_{1}} \times {_{1} I_{1}} {_{1} {\omega}_{1}} + \sum_{i=1}^{4}{({_{1}{\omega}_{1}} \times e_z) \cdot I_{M} {_{M_i}\omega_{M_i}} } \\
    %&= \begin{bmatrix} 0              & d \cdot k_{T} & 0             & -d \cdot k_{T} \\ 
                       %-d \cdot k_{T} & 0             & d \cdot k_{T} & 0  \\
                       %-k_{drag}      & k_{drag}      & -k_{drag}    & k_{drag}
       %\end{bmatrix}
    %\cdot 
    %\begin{bmatrix}
      %{_{M_1} {\omega}^2_{M_1}} \\
      %{_{M_2} {\omega}^2_{M_2}}\\
      %{_{M_3} {\omega}^2_{M_3}}\\
      %{_{M_4} {\omega}^2_{M_4}}
    %\end{bmatrix}
%\end{align}
%\subsubsection{Quaternionen}\label{subsub:Quaternionen}
%Mit $\dot q = \frac{1}{2} q \otimes \begin{bmatrix} 0 \\ \omega \end{bmatrix} $ gilt:
%
%\begin{align}
    %{_{0}\dot{r}_{1}} &= \left[ q \otimes {_{1}v_{1}} \otimes \overline{q} \right]_{\left[1:3\right]} \\
    %m_{B_1} \frac{d_1}{dt} {_{1}v_{1}} &+ {_{1}\omega_{1}} \times m_{B_1} {_{1}v_{1}} = m_{B_1} \left[\overline{q} \otimes \begin{pmatrix} 0 \\ 0 \\ -g \end{pmatrix} \otimes q \right]_{\left[1:3\right]} + {\begin{pmatrix} 0 \\ 0 \\ \sum_{i = 1}^4 k_{T} {_{M_i}\omega^2_{1, M_i}} \end{pmatrix}} \\
    %\dot q &= \frac{1}{2} q \otimes \begin{bmatrix} 0 \\ \omega \end{bmatrix} \\
    %{_{1} I_{1}} {_{1} \dot{\omega}_{1}} &+ {_{1} {\omega}_{1}} \times {_{1} I_{1}} {_{1} {\omega}_{1}} + \sum_{i=1}^{4}{({_{1}{\omega}_{1}} \times e_z) \cdot I_{M} {_{M_i}\omega_{M_i}} } \\
    %&= \begin{bmatrix} 0              & d \cdot k_{T} & 0             & -d \cdot k_{T} \\ 
                       %-d \cdot k_{T} & 0             & d \cdot k_{T} & 0  \\
                       %-k_{drag}      & k_{drag}      & -k_{drag}    & k_{drag}
       %\end{bmatrix}
    %\cdot 
    %\begin{bmatrix}
      %{_{M_1} {\omega}^2_{M_1}} \\
      %{_{M_1} {\omega}^2_{M_2}}\\
      %{_{M_1} {\omega}^2_{M_3}}\\
      %{_{M_1} {\omega}^2_{M_4}}
    %\end{bmatrix}
%\end{align}
%
%
%\subsubsection{Rotationsmatrix}\label{subsub:Rotationsmatrix}
%Mit $q = [q_0, q_1, q_2, q_3]^T$, $\dot q = \frac{1}{2} q \otimes \begin{bmatrix} 0 \\ \omega \end{bmatrix} $, $q_0^2 + q_1^2 + q_2^2 + q_3^2 = 1$ und \\
%$R_{0, 1}(q) = \left[ \begin{matrix} 1-2(q_2^2 + q_3^2) &
%-2q_0q_3+2q_1q_2 &
%2q_0q_2+2q_1q_3 \\
%
%2q_0q_3+2q_1q_2 &
%1-2(q_1^2 + q_3^2) &
%-2q_0q_1+2q_2q_3 \\
%
%-2q_0q_2+2q_1q_3 &
%2q_0q_1+2q_2q_3 &
%1-2(q_1^2 + q_2^2)
%\end{matrix}
%\right]
%$ 
%
%\begin{align}
    %{_{0}\dot{r}_{1}} &= R_{0, 1}(q) {_{1}v_{1}} \\
    %m_{B_1} \frac{d_1}{dt} {_{1}v_{1}} &+ {_{1}\omega_{1}} \times m_{B_1} {_{1}v_{1}} = R(q)^T \begin{pmatrix} 0 \\ 0 \\ -m_{B_1} \cdot g \end{pmatrix} + {\begin{pmatrix} 0 \\ 0 \\ \sum_{i = 1}^4 k_{T} {_{M_i}\omega^2_{1, M_i}} \end{pmatrix}} \\
    %\dot q &= \frac{1}{2} q \otimes \begin{bmatrix} 0 \\ \omega \end{bmatrix} \\
    %{_{1} I_{1}} {_{1} \dot{\omega}_{1}} &+ {_{1} {\omega}_{1}} \times {_{1} I_{1}} {_{1} {\omega}_{1}} + \sum_{i=1}^{4}{({_{1}{\omega}_{1}} \times e_z) \cdot I_{M} {_{M_i}\omega_{M_i}} } \\
    %&= \begin{bmatrix} 0              & d \cdot k_{T} & 0             & -d \cdot k_{T} \\ 
                       %-d \cdot k_{T} & 0             & d \cdot k_{T} & 0  \\
                       %-k_{drag}      & k_{drag}      & -k_{drag}    & k_{drag}
       %\end{bmatrix}
    %\cdot 
    %\begin{bmatrix}
      %{_{M_1} {\omega}^2_{M_1}} \\
      %{_{M_1} {\omega}^2_{M_2}}\\
      %{_{M_1} {\omega}^2_{M_3}}\\
      %{_{M_1} {\omega}^2_{M_4}}
    %\end{bmatrix}
%\end{align}
\subsection{fehlt noch}

\begin{itemize}
	\item Literatur
	\item Quaternionen-Multiplikation
	\item Herleitung
\end{itemize}

\subsection{Variablen und Konstanten des Quadcopters}

\textbf{Koordinatensysteme}
Zu vereinfachten Betrachtung bestimmter Variablen werden in diesem Modell drei (bzw. sechs) verschiedene Koordinatensysteme verwendet:

\begin{enumerate}
	\item Inertialsystem $ \mathcal{I} $: Indizierung mit "`0"', unbewegtes Beobachtersystem
	\item Körpersystem $ \mathcal{B}_1 $: Indizierung mit "`1"', alle "`nicht drehenden"' Teile des Quadcopters
	\item vier Rotorsysteme $ \mathcal{B}_{M_i}, i=\{1,2,3,4\} $: Indizierung mit $ M_i $, z-Achse des Koordinatensystems entspricht der Drehachse des jeweiligen Rotors
\end{enumerate}

\textbf{Konstanten}

\begin{itemize}
	\item $ m_{ges} $: Gesamtmasse des Quadcopters
	\item $ _{1}\mathbf{I}_{1} =
		\begin{bmatrix} I_{xx} & 0 & 0 \\
										0 & I_{yy} & 0 \\
										0 & 0 & I_{zz}
		\end{bmatrix} $: Trägheitsmatrix des Quadcopters bzgl. des Massenschwerpunkts im Körpersystem
		\item $ I_{M} $: Trägheitsmatrix eines Motors (im Rotorsystem)
		\item $ g $: Erdbeschleunigung
		\item $ d $: Abstand der Drehachse eines Rotors zum Massenschwepunkt des Copters
		\item $ k_T=C_T\rho_{Luft}A_{Rotor}r^2 $: Schubkonstante eines Motors; $ \rho_{Luft} $ ist die Luftdichte, $ C_T $ der Schubkoeffizient des Motors, $ A_{Rotor} $ die Fläche, die der Rotor bei einer Umdrehung überstreicht und $ r $ der Radius des Rotors
		\item $ k_{drag} = \frac{1}{2}\rho_{Luft}A_{Rotor}r^2 $: Luftwiderstandskoeffizient eines Motors
\end{itemize}

\textbf{Variablen}

\begin{itemize}
	\item \( _{0}r_{1} \in \R^3 \): Position des Copters im Inertialsystem
	\item \( q=( q_1, q_2, q_3, q_4) \in \R^4 \): Quaternion, welches die Drehung des Quadcopters bzgl. des Inertialsystems angibt
	\item \( _{1}v_{1} \): Geschwindigkeit des Copters im Körpersystem
	\item \( _{1}\omega_{1} \): Winkelgeschwindigkeit des Copters im Körpersystem
	\item \( _{M_i}\omega_{1,M_1} \): Winkelgeschwindigkeit der einzelnen Rotoren; es gilt immer \( \omega_{M_1} > 0 \), die Vorzeichenänderung aufgrund der unterschiedlichen Drehrichtung der Rotoren wird direkt in den Gleichungen eingesetzt
	\item $ R=R_{0,1}= \begin{bmatrix}
														 1-2(q_3^2 + q_4^2) & -2q_1q_4 +2q_2q_3 & 2q_1 q_3+2q_2q_4 \\

														 2q_1q_4+2q_2q_3 & 1-2(q_2^2 + q_4^2) & -2q_1q_2+2q_3q_4 \\

														 -2q_1q_3+2q_2q_4 & 2q_1q_2+2q_3q_4 & 1-2(q_2^2 + q_3^2)
											\end{bmatrix}
				$: Drehmatrix vom Körpersystem ins Inertialsystem; es gilt $ R^{-1} = R^T $
\end{itemize}


\subsection{Newton-Euler Gleichungen}

Es seien: \\
$ \theta := \left[ \begin{matrix} {_{0}} r_{1}, & q \end{matrix} \right]^T \in \R^{7} $ \\
$ \dot \theta := \left[ \begin{matrix} {_{1}} v_{1},&{_{1}} \omega_{1} \end{matrix}\right]^T \in \R^6 $ \\
$ \ddot \theta := \left[ \begin{matrix} {_{1}}\dot v_{1},&{_{1}}\dot \omega_{1} \end{matrix} \right]^T \in \R^6 $ \\
$ \omega_{M} = \left[\begin{matrix} _{M_1}\omega_{1,M_1}, & _{M_2}\omega_{1,M_2}, & _{M_3}\omega_{1,M_3}, & _{M_44}\omega_{1,M_4} \end{matrix}\right]^T \in \R^4$ \\

Die Dynamik des Quadcopters kann somit als folgende Differentialgleichung dargestellt werden:

\begin{align*}
  M(\theta) \ddot\theta + \Theta (\theta, \dot \theta) = T(\dot \theta, \omega_M(t))
\end{align*}

mit 

\begin{align*}
	M(\theta(t)) &= 
		\left[ 
				\begin{array}{c@{}c@{}}
						m_{ges} \cdot \mathbf{E_{3x3}}  & \quad \mathbf{0} \\
						\mathbf{0} & \quad \mathbf{{_{1}I_{1}}} \\
				\end{array}
		\right] \in \R^{6x6}
\end{align*}

Hierbei ist $ \mathbf{E_{3x3}} $ die Einheitsmatrix in $ \R^3 $. \\
Man beachte, dass $ M $ gar nicht von $ t $ abhängt, sondern für alle Zeiten konstant ist. Außerdem ist $ M $ eine Diagonalmatrix.
	
\begin{align*}	
	\Theta(\theta, \dot \theta ) &=
			\left[
				\begin{matrix}
					m_{ges} \left( {_{1}} \omega_{1, y} \cdot {_{1}} v_{1, z} -  {_{1}} \omega_{1, z} \cdot {_{1}} v_{1, y} + R_{1, 3}^T(q) \cdot g \right)\\
					m_{ges} \left( {_{1}} \omega_{1, z} \cdot {_{1}} v_{1, x} -  {_{1}} \omega_{1, x} \cdot {_{1}} v_{1, z} + R_{2, 3}^T(q) \cdot g \right)\\
					m_{ges} \left( {_{1}} \omega_{1, x} \cdot {_{1}} v_{1, y} -  {_{1}} \omega_{1, y} \cdot {_{1}} v_{1, x} + R_{3, 3}^T(q) \cdot g \right)\\
					_{1}\omega_{1,y} \cdot _{1}\omega_{1,z}\left(I_{zz}-I_{yy}\right)  \\
					_{1}\omega_{1,z} \cdot _{1}\omega_{1,z}\left(I_{xx}-I_{zz}\right)  \\
					_{1}\omega_{1,x} \cdot _{1}\omega_{1,y}\left(I_{yy}-I_{xx}\right)  \\
				\end{matrix}
			\right] \\
\end{align*}

\begin{align*}		
	T(\dot\theta, \omega_M(t))  &= 
	\left[
	\begin{matrix}   0 \\ 
	                 0 \\ 
                   \sum_{i = 1}^4 k_{T} _{M_i}\omega^2_{1,M_i} \\
                   I_M _{1}\omega_{1, y} (- _{M_1}\omega_{1,M_1}+ _{M_2}\omega_{1,M_2} - _{M_3}\omega_{1,M_3} + _{M_4}\omega_{1,M_4}) + d k_{T} \left({_{M_2}\omega}^2_{1,M_2} - {_{M_4}\omega}^2_{1,M_4}\right) \\
                   -I_M {_{1}} \omega_{1, x} ( -_{M_1}\omega_{1,M_1} + _{M_2}\omega_{1,M_2} - _{M_3}\omega_{1,M_3} + _{M_4}\omega_{1,M_4}) +d k_{T} \left({_{M_3}\omega}^2_{1,M_3} + {_{M_1}\omega}^2_{M_1}\right) \\
                  k_{drag} \cdot \left(- {_{M_1}\omega}^2_{1,M_1} + {_{M_2}\omega}^2_{1,M_2}  - {_{M_3}\omega}^2_{1,M_3} +  {_{M_4}\omega}^2_{1,M_4}\right)  \end{matrix}
	\right]
\end{align*}



Für die optimal control Formulierung führe zusätzlich ein:
\begin{itemize}
	\item state: $x(t) =\left(\begin{matrix} \theta, &\dot \theta \end{matrix}\right)^T \in \R^{13}$
	\item control: $u(t) = \omega_{M} \in \R^{4}$
\end{itemize}

Die ODE-Nebenbedingung für das Optimalsteuerungsproblem hat damit die Form:
\begin{align}
\frac{d}{dt} 
  \left(
      \begin{matrix}
          x_{0:2}(t) \\
          x_{3:6}(t) \\
          x_{7:12}(t)  
      \end{matrix}
  \right)
  = 
  \left(
      \begin{matrix}
          \mathbf{R_{01}}(q)x_{7:9} \\
          \frac{1}{2} x_{3:6} \otimes \left(\begin{matrix} 0, & x_{10:12} \end{matrix}\right)^T\\
          M^{-1} (T(u(t), x(t)) - \Theta(x(t))
      \end{matrix}
  \right)
\end{align}

Dabei steht $ \otimes $ für die Quaternionenmultiplikation. 

\subsubsection{Jacobian - und Hessematrix für $\Theta$}

\end{document}