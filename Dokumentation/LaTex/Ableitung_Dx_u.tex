\documentclass[12pt, a4paper]{scrartcl}
\usepackage{fullpage}
\usepackage[latin1]{inputenc} %Zeichensatzkodierung (auch Sonderzeichen -> �,�,�)
\usepackage[english, ngerman]{babel} %Sprache -> Silbentrennung, automatisch generierte Texte auf deutsch; ngerman -> NEUE deutsche Rechtschreibung
\usepackage[T1]{fontenc} %Sonderzeichen allg
\usepackage{lmodern} %PDF-optimierte Schrift

\usepackage{amsmath, amssymb, amsbsy, amsthm} %Mathepakete ; amsthm: Theorem-Umgebung; muss nach amsmath eingebunden werden

\usepackage{xcolor} %bunte Schrift 

\setlength{\parindent}{0pt}

\newcommand{\R}{\mathbb{R}}


\begin{document}

\section{Motivation}
Ziel: Berechnung der Matrizen \( Dx_{x_{i-1}} \) und \(  Dx_{u_{i-1}} \) ohne \(x\) zu kennen?

\section{Vorgehen}

\textcolor{red}{Bedingungen pr�fen, erf�llen unsere Gleichungen das? Erf�llen sie es auch mit numerischen L�sungen?} \\

Gegeben sei die Differentialgleichung:
\[ \dot{x}(t)=f(x(t), u(t)) \]

Idee: Anstatt zuerst die Differentialgleichung zu l�sen, leite diese nach \(x\) bzw. \(u\) ab und l�se erst dann die Differentialgleichung: (Damit das nachfolgende Vorgehen �berhaupt erlaubt, bzw. alle Ausdr�cke wohldefiniert sind, muss \(x(t)\) in allen Variablen stetig differenzierbar sein, damit man den Satz von Schwarz anwenden kann.) \\
Am Computer muss dieses Verfahren iterativ f�r jeden Zeitschritt angewendet werden. D.h. in jedem neuen Zeitschritt berechnet man \(M\) und \(N\) neu (also sozusagen \(M_{i-1}, N_{i-1}\)) \textcolor{red}{Warum \(i-1\)? Ist das jetzt zum Zeitpunkt \(i\) oder zu einem anderen?} \\
\\
Man benutzt also:

\begin{align*}
	\frac{\partial }{\partial x_{i-1}} \left( \frac{\textup{d} }{\textup{d} t} x \right) & = \frac{\partial }{\partial x_{i-1}} \left( f(x, u) \right)  \\
	\vspace{1em} \\
	\Leftrightarrow \qquad \frac{\textup{d} }{\textup{d} t} \underbrace{\left( \frac{\partial x}{\partial x_{i-1}} \right)}_{=:M_{i-1}} & = \frac{\partial }{\partial x} f(x, u) \cdot \underbrace{\frac{\partial x}{\partial x_{i-1}}}_{=:M_{i-1}} \\	
\end{align*}

Und:

\begin{align*}
	\frac{\partial }{\partial u_{i-1}} \left( \frac{\textup{d} }{\textup{d} t} x \right) & = \frac{\partial }{\partial u_{i-1}} \left( f(x, u) \right)  \\
	\vspace{1em} \\
	\frac{\textup{d} }{\textup{d} t} \underbrace{\left( \frac{\partial x}{\partial u_{i-1}} \right)}_{=:N_{i-1}} & = \frac{\partial }{\partial u} f(x, u) \cdot \underbrace{\frac{\partial x}{\partial u_{i-1}}}_{=:N_{i-1}} \\	
\end{align*}

\textcolor{red}{Welche Dimension haben M und N??? Wo sind die anderen Partiellen Ableitungen (\(\frac{\partial u}{\partial x/u_{i-1}}\))?} \\

Es kann nun also folgendes System gel�st werden, um die Ableitung nach \(x_{i-1}\) bzw. \(u_{i-1}\) zu berechnen:

\[ \frac{\textup{d}}{\textup{d} t} \begin{bmatrix} x \\ M_{i-1} \\ N_{i-1} \end{bmatrix}  = \begin{bmatrix} f(x,u) & f_{x}(x,u) & f_{u}(x,u) \end{bmatrix} \cdot \begin{bmatrix} 1 \\M_{i-1} \\ N_{i-1} \end{bmatrix}\]

\textcolor{red}{Achtung auf Multiplikation!!! Dimensionen passen? Matrizen? Vektoren?} \\
Diese Differentialgleichung kann nun numerisch gel�st werden.
%\textcolor{red}{Ist das ein Randwertproblem? nur mit zb. Euler oder braucht man multiple shooting? Anfangsbedingungen unten?}

Setzt man den Zeitpunkt \(t_{i-1}\) ein, so erh�lt man f�r \(M_{i-1}\) und \(N_{i-1}\) die Anfangsbedingungen der ODE:
\[ M(t_{i-1})= \frac{\partial x(t_{i-1})}{\partial x_{i-1}} = \mathbb{I}\]
\[ N(t_{i-1})= \frac{\partial x(t_{i-1})}{\partial u_{i-1}} = 0 \]

%\textcolor{red}{Zusammenhang mit multiple shooting Constraint \(x(t_{i})-s_{i}=0\) mit \(s_{i}\) gesch�tzter Anfangswert im neuen Intervall? }

\section{Voraussetzungen...}

\subsection{...f�r das SQP-Verfahren}

\begin{itemize}
	\item am besten Haupts�chlich Gleichungsnebenbedingungen
	\item Zielfunktion und Nebenbedingungen zweimal stetig differenzierbar
\end{itemize}

\section{Einzelne L�sungsschritte}

\begin{enumerate}
	\item SQP-Verfahren
	\begin{enumerate}
		\item normales Optimierungsverfahren \(\rightarrow\) Dynamik muss schon aufgel�st sein
		\item innerhalb noch mal explizites L�sungsverfahren (zB Newton, Newton-Rhapson?)
		\item Approximation der Hessematrix \(\nabla^2_{xx}L(x^k,\mu^k)\) m�glich \textcolor{red}{(Machen wir nicht?)}
		\item haupts�chlich f�r Gleichungsnebenbedingungen, aber auch f�r Ungleichungsnebenbedingungen m�glich
		\item Globalisierung mit Penalty-Funktion m�glich
		\item Schwierigkeiten: m�gliche Unzul�ssigkeit der Teilprobleme (nur bein Ungleichungsnebenbedingungen?)
	\end{enumerate}
	\item L�sung der Dynamik
	\begin{enumerate}
		\item 
	\end{enumerate}
\end{enumerate}

\subsection{Genauere Berachtung der Hessematrix der Lagrangefunktion}

\textbf{Lagrangefunktion}

\[ L: \R^n \times (\R^m) \times \R^p, \quad L(x,\lambda,\mu)= f(x)+(\lambda^{T}g(x))+\mu^{T}h(x)\]
\(f(x)\): Zielfunktion \\
\(h(x)\): Gleichungsnebenbedingungen

\textcolor{red}{Achtung: Optimierung bei uns nach zwei Variablen \(x\) und \(u\), diese stecken hier beide in \(x\)!} \\

\textbf{Hessematrix f�r SQP-Verfahren}

\[ H_{k}=\nabla^2_{xx}L(x^k,\mu^k)=\nabla^2_{xx}f(x^k)+(\mu^k)^T\nabla^2_{xx}h(x^k) \]

Hier wird die Hessematrix der Nebenbedingungen gebraucht \(\rightarrow\) auch die Nebenbedingungen aus Multiple shooting

 

\end{document}